\chapter{进入保护模式} \label{CHpm}

前面我们看到,通过一些很简单的代码,我们做到了启动一个微型系统,加载文件系统中的文件进入内存并运行的功能。应该注意的是,在前面的代码中我们使用的内存空间都很小。我们看一下~boot.bin~和~LOADER.BIN~的大小就能感觉出来(当然,可执行文件小未必使用内存空间小,但是这两个文件也太小了~\smiley)。

\begin{Command}
$ ls -l boot.bin LOADER.BIN 
-rwxr-xr-x 1 solrex solrex 512 2008-04-26 16:34 boot.bin
-rwxr-xr-x 1 solrex solrex  15 2008-04-26 16:34 LOADER.BIN
\end{Command}

boot.bin~是~512~个字节(其中还有我们填充的内容,实际指令只有~480~个字节),而~LOADER.BIN~更过分,只有~15~个字节大小。可想而知这两个文件在内存中能使用多大的空间吧。如果读者有些汇编语言经验的话,就会发现我们在前面的程序中使用的存储器寻址都是在实模式下进行的,即:由段寄存器(cs,~ds:~16-bit)配合段内偏移地址(16-bit)来定位一个实际的~20-bit~物理地址,所以我们前面的程序最多支持~$2^{20} = 2^{10}*2^{10} = 1024*1024$ bytes = 1MB~的寻址空间。

哇,~1MB~不小了,我们的操作系统加一起连~1KB~都用不到,~1MB~寻址空间足够了。但是需要考虑到的一点是,就拿我们现在用的~1.44MB的(已经被淘汰的)软盘标准来说,如果软盘上某个文件超过~1MB~,我们的操作系统就没办法处理了。那么如果以后把操作系统安装到硬盘上之后呢?我们就没办法处理稍微大一点的文件了。

所以我们要从最原始的~Intel 8086/8088 CPU~的实模式中跳出来,进入~Intel 80286~之后系列~CPU~给我们提供的保护模式。这还将为我们带来其它更多的好处,具体内容请继续往下读。

\section{实模式和保护模式}

\BOXED{0.9\textwidth}{
\danger\\ 如果您需要更详细的知识,也许您更愿意去读 Intel 的手册,本节内容主要集中在:\href{http://download.intel.com/design/processor/manuals/253668.pdf}{Intel\textregistered~64 and IA-32 Architectures Software Developer's Manual, Volume 3A: System Programming Guide}, 第~2~章和第~3~章.\enddanger
}

\subsection{一段历史}

Intel~公司在~1978~年发布了一款~16~位字长~CPU: 8086~,最高主频~5 MHz$\sim$10 MHz~,集成了~29,000~个晶体管,这款在今天感觉像玩具一样的~CPU~却是奠定今天~Intel PC~芯片市场地位的最重要的产品之一。虽然它的后继者~8088~,加强版的~8086~(增加了一个~8~比特的外部总线)才是事实上的~IBM~兼容机(PC,个人电脑)雏形的核心,但人们仍然习惯于用~8086~作为厂商标志代表~Intel~。

因为受到字长(16~位)的限制,如果仅仅使用单个寄存器寻址,~8086~仅仅能访问~64KB($2^{16}$)~的地址空间,这显然不能满足一般要求,而当时~1MB($2^{20}$)~对于一般的应用就比较足够了,所以~8086~使用了~20~位的地址线。

在~8086~刚发布的时候,没有“实模式”这个说法,因为当时的~Intel CPU~只有一种模式。在~Intel~以后的发布中,~80286~引入了“保护模式”寻址方式,将~CPU~的寻址范围扩大到~16($2^{24}$) MB~,但是~80286~仍然是一款~16~位~CPU~,这就限制了它的广泛应用。但是“实模式”这个说法,就从~80286~开始了。

接下来的发展就更快了,1985~年发布的~i386~首先让~PC CPU~进入了~32~位时代,由此而带来的好处显而易见,寻址能力大大增强,但是多任务处理和虚拟存储器的需求仍然推动着~i386~向更完善的保护模式发展。下面我们来了解一下“实模式”和“保护模式”的具体涵义。

\subsection{实模式}

实模式(real mode),有时候也被成为实地址模式(real address mode)或者兼容模式(compatibility mode)是~Intel 8086 CPU~以及以其为基础发展起来的~x86~兼容~CPU~采用的一种操作模式。其主要的特点有:20~比特的分段访问的内存地址空间(即~1~MB~的寻址能力);程序可直接访问~BIOS~中断和外设;硬件层不支持任何内存保护或者多任务处理。~80286~之后所有~x86 CPU~在加电自举时都是首先进入实模式;~80186~以及之前的~CPU~只有一种操作模式,相当于实模式。

\subsection{保护模式}

保护模式(protected mode),有时候也被成为保护的虚拟地址模式(protected virtual address mode),也是一种~x86~兼容~CPU~的工作模式。保护模式为系统软件实现虚拟内存、分页机制、安全的多任务处理的功能支持,还有其它为操作系统提供的对应用程序的控制功能支持,比如:特权级、实模式应用程序兼容、虚拟~8086~模式。

\subsection{实模式和保护模式的寻址模式}

前面提到过,实模式下的地址线是~20~位的,所以实模式下的寻址模式使用分段方式来解决~16~位字长机器提供~20~位地址空间的问题。这个分段方法需要程序员在编制程序的过程中将存储器划分成段,每个段内的地址空间是线性增长的,最大可达~64K($2^16$),这样段內地址就可以使用~16~位表示。段基址(~20-bit~)的最低~4~位必须是~0~,这样段基址就可以使用~16~位段地址来表示,需要时将段地址左移~4~位就得到段起始地址。除了便于寻址之外,分段还有一个好处,就是将程序的代码段、数据段和堆栈段等隔离开,避免相互之间产生干扰。

当计算某个单元的物理地址时,比如汇编语言中的一个~Label~,就通过段地址(~16-bit~)左移~4~位得到段基址(~20-bit~),再加上该单元(~Label~)的段內偏移量(~16-bit~)来得到其物理地址(~20-bit~),如图~\ref{rm_addr}~所示。

\begin{figure*}[!t]
\centerline{\subfloat[实模式寻址模型]{\includegraphics[width=.48\textwidth,keepaspectratio]{rm_addr}%
\label{rm_addr}}
\hfil
\subfloat[保护模式寻址模型]{\includegraphics[width=.48\textwidth,keepaspectratio]{protected_seg}%
\label{protected_seg}}}
\caption{实模式与保护模式寻址模型比较}
\label{real_vs_pro}
\end{figure*}

一般情况下,段地址会被放在四个段寄存器中,即:代码段~CS,数据段~DS,堆栈段~SS~和附加段~ES~寄存器。这样在加载数据或者控制程序运行的时候,只需要一个偏移量参数,CPU~会自动用对应段的起始地址加上偏移量参数来得到需要的地址。(后继~CPU~又加上了两个段寄存器~FS~和~GS~,不过使用方式是基本一样的。)

由此可见,实模式的寻址模式是很简单的,就是用两个~16~位逻辑地址(段地址:偏移地址)组合成一个~20~位物理地址,而保护模式的寻址方式就要稍微复杂一点了。

\BOXED{0.9\textwidth}{
Intel~的~CPU~在保护模式下是可以选择打开分页机制的,但为了简单起见,我们先不开启分页机制,所以下面的讲解针对只有分段机制的保护模式展开。
}

在保护模式下,每个单元的物理地址仍然是由逻辑地址表示,但是这个逻辑地址不再由(段地址:偏移地址)组成了,而是由(段选择子:偏移地址)表示。这里的偏移地址也变成了~32~位的,所以段空间也比实模式下大得多。偏移地址的意思和实模式下并没有本质不同,但段地址的计算就要复杂一些了,如图~\ref{protected_seg}~所示。段基址(Segment Base Address)被存放在段描述符(Segment Descriptor)中,GDT(Global Descriptor Table,全局段选择子表)是保存着所有段选择子的信息,段选择子(Segment Selector)是一个指向某个段选择子的索引。

如图~\ref{protected_seg}~所示,当我们计算某个单元的物理地址时,只需要给出(段选择子:偏移地址),CPU~会从~GDT~中按照段选择子找到对应的段描述符,从段描述符中找出段基址,将段基址加上偏移量,就得到了该单元的物理地址。

\section{与保护模式初次会面}

介绍完了保护模式和实模式的不同,下面我们就尝试一下进入保护模式吧。在上一章我们已经实现了用启动扇区加载引导文件,所以这里我们就不用再去管启动扇区的事情了,下面的修改均在~loader.S~中进行。上一章的~loader.S~仅仅实现在屏幕的上方中间打印了一个~\code{L}~,下面我们的~loader.S~要进入保护模式来打印一些新东西。

首先,我们来理清一下应该如何进入保护模式:

\begin{enumerate}
  \item 我们需要一个~GDT。由于保护模式的寻址方式是基于~GDT~的,我们得自己写一个~GDT~数据结构并将其载入到系统中。
  \item 我们需要为进入保护模式作准备。由于保护模式和实模式运行方式不同,在进入保护模式之前,我们需要一些准备工作。
  \item 我们需要一段能在保护模式下运行的代码 demo,以提示我们成功进入了保护模式。
\end{enumerate}

下面我们就来一步步完成我们的第一个保护模式~loader~。

\subsection{GDT~数据结构}

要写~GDT,首先得了解~GDT~的数据结构。GDT~实际上只是一个存储段描述符的线性表(可以理解成一个段描述符数组),对它的要求是其第一个段描述符置为空,因为处理机不会去处理第一个段描述符,所以理解~GDT~的数据结构难点主要在于理解段描述符的数据结构。

段描述符主要用来为处理机提供段位址,段访问控制和状态信息。图~\ref{seg_desc}~显示了一个基本的段描述符结构:

\FIG{段描述符}{seg_desc}{.9\textwidth}

看到上面那么多内容,是不是感觉有点儿恐怖啊!其实简单的来看,我们现在最关注的是段基址,就是图~\ref{seg_desc}~中标记为~Base~的部分。可以看到,段基址在段描述符中被分割为三段存储,分别是:Base 31:24, Base 23:16, Base Address 15:0,把这三段拼起来,我们就得到了一个~32~位的段基址。

有了段基址,就需要有一个界限来避免程序跑丢发生段错误,这个界限就是图~\ref{seg_desc}~中标记为~Limit~的部分,将~Seg. Limit 19:16~和~Segment Limit 15:0~拼起来我们就得到了一个~20~位的段界限,这个界限就是应该是段需要的长度了。

下面还要说的就是那个~D/B Flag~,D/B~代表~Default Operation Size~,0~代表~16~位的段,1~代表~32~位的段。为了充分利用~CPU~,我们当然要设置为~32~位模式了。剩下那些乱七八糟的~Flag~呢,无非就是提供段的属性(代码段还是数据段?只读还是读写?),我们将在第~\ref{CHpm_desattr}~节为大家详细介绍。

这些东西那么乱,难道要每次一点儿一点儿地计算吗?放心,程序员自有办法,请看下面的程序:

\VerbatimInput[fontfamily=tt,fontsize=\footnotesize,frame=lines, framerule=0.4mm, numbers=left, numbersep=3pt, tabsize=2, firstline=56, lastline=70]{../src/chapter3/1/pm.h}
\codecaption{自动生成段描述符的宏定义(节自 chapter3/1/pm.h)}\label{CHpm_descm}

图~\ref{CHpm_descm}~中所示,就是自动生成段描述符的汇编宏定义。我们只需要给宏~Descriptor~三个参数:Base(段基址), Limit(段界限[段长度]), Attr(段属性),Descriptor~就会自动将三者展开放到段描述符中对应的位置。看看我们在程序中怎么使用这个宏:

\VerbatimInput[fontfamily=tt,fontsize=\footnotesize,frame=lines, framerule=0.4mm, numbers=left, numbersep=3pt, tabsize=2, firstline=21, lastline=26]{../src/chapter3/1/loader.S}
\codecaption{自动生成段描述符的宏使用示例(节自 chapter3/1/loader.S)}\label{CHpm_descmu}

图~\ref{CHpm_descmu}~中,就利用~Descriptor~宏生成了三个段描述符,形成了一个~GDT。注意到没有,第一个段描述符是空的(参数全为~0)。这里~LABEL\_DESC\_CODE32~的段基址为~0~是因为我们无法确定它的准确位置,它将在运行期被填入。

有人可能会产生疑问,段基址和段界限什么意思我们都知道了,那段属性怎么回事呢?~DA\_C, DA\_32, DA\_DRW~都是什么东西啊?是这样的,为了避免手动一个一个置段描述符中的~Flag~,我们预先定义了一些常用属性,用的时候只需要将这些属性加起来作为宏~Descriptor~的参数,就能将段描述符中的所有~flag~置上(记得~C~语言中~fopen~的参数吗?)。这些属性的定义如下(没必要细看,用的时候再找即可):

\VerbatimInput[fontfamily=tt,fontsize=\footnotesize,frame=lines, framerule=0.4mm, numbers=left, numbersep=3pt, tabsize=2, firstline=11, lastline=55]{../src/chapter3/1/pm.h}
\codecaption{预先设置的段属性(节自 chapter3/1/pm.h)}\label{CHpm_segattr}

\subsection{保护模式下的~demo}

为什么把这节提前到第~\ref{CHpm_secloadgdt}~节前讲呢?因为要写入~GDT~正确的段描述符,首先要知道段的信息,我们就得先准备好这个段:

\VerbatimInput[fontfamily=tt,fontsize=\footnotesize,frame=lines, framerule=0.4mm, numbers=left, numbersep=3pt, tabsize=2, firstline=83, lastline=99]{../src/chapter3/1/loader.S}
\codecaption{第一个在保护模式下运行的 demo(节自 chapter3/1/loader.S)}\label{CHpm_demo1}

其实这个段的作用很简单,通过操纵视频段数据,在屏幕中间打印一个红色的"P"(和我们前面使用~BIOS~中断来打印字符的方式有所不同)。

\subsection{加载~GDT} \label{CHpm_secloadgdt}

GDT~所需要的信息我们都知道了,GDT~表也通过图~\ref{CHpm_descmu}~中的代码实现了。那么,我们应该向~GDT~中填入缺少的信息,然后载入~GDT~了。将~GDT~载入处理机是用~\code{lgdt}~汇编指令实现的,但是~\code{lgdt}~指令需要存放~GDT~的基址和界限的指针作参数,所以我们还需要知道~GDT~的位置和~GDT~的界限:

\VerbatimInput[fontfamily=tt,fontsize=\footnotesize,frame=lines, framerule=0.4mm, numbers=left, numbersep=3pt, tabsize=2, firstline=22, lastline=63]{../src/chapter3/1/loader.S}
\codecaption{加载~GDT(节自 chapter3/1/loader.S)}\label{CHpm_loadgdt}

图~\ref{CHpm_loadgdt}~中~\code{GdtPtr}~所指,即为~GDT~的界限和基址所存放位置。某段描述符对应的~GDT~选择子,就是其段描述符相对于~GDT~基址的索引(在我们例子里~GDT~基址为~\code{LABEL\_GDT}~指向的位置)。这里需要注意的是,虽然我们在代码中写:

\begin{Command}
.set SelectorCode32, (LABEL_DESC_CODE32 - LABEL_GDT)
\end{Command}
但实际上段选择子在使用时需要右移~3~个位作为索引去寻找其对应的段描述符,段选择子的右侧~3~个位是为了标识~TI~和~RPL~的,如图~\ref{seg_selector}~所示,这点我们将在第~\ref{CHpm_ldt}~节和第~\ref{CHpm_desattr}~节中详细介绍。但是这里为什么能直接用地址相减得到段选择子呢?因为段选择子的大小是~8 bit~,用地址相减的话,最右侧三个位就默认置~0~了。

在图~\ref{CHpm_loadgdt}~中所示的代码,主要干了两件事:第一,将图~\ref{CHpm_demo1}~所示~demo~的段基址放入~GDT~中对应的段描述符中;第二,将~GDT~的基址放到~\code{GdtPtr}~所指的数据结构中,并加载~\code{GdtPtr}~所指的数据结构到~GDTR~寄存器中(使用~\code{lgdt}~指令)。

\subsection{进入保护模式}

进入保护模式前,我们需要将中断关掉,因为保护模式下中断处理的机制和实模式是不一样的,不关掉中断可能带来麻烦。使用~\code{cli}~汇编指令可以清除所有中断~flag。

由于实模式下仅有~20~条地址线:A0, A1, \ldots, A19,所以当我们要进入保护模式时,需要打开~A20~地址线。打开~A20~地址线有至少三种方法,我们这里采用~IBM~使用的方法,通常被称为:“Fast A20 Gate”,即修改系统控制端口~92h~,因为其端口的第~1~位控制着~A20~地址线,所以我们只需要将~0b00000010~赋给端口~92h~即可。

当前面两项工作完成后,我们就可以进入保护模式了。方法很简单,将~cr0~寄存器的第~0~位~PE~位置为~1~即可使~CPU~切换到保护模式下运行。

\VerbatimInput[fontfamily=tt,fontsize=\footnotesize,frame=lines, framerule=0.4mm, numbers=left, numbersep=3pt, tabsize=2, firstline=64, lastline=77]{../src/chapter3/1/loader.S}
\codecaption{进入保护模式(节自 chapter3/1/loader.S)}\label{CHpm_enablepm}

\subsection{特别的混合跳转指令}

虽然已经进入了保护模式,但由于我们的~CS~寄存器存放的仍然是实模式下~16~位的段信息,要跳转到我们的~demo~程序并不是那么简单的事情。因为~demo~程序是~32~位的指令,而我们现在仍然运行的是~16~位的指令。从~16~位的代码段中跳转到~32~位的代码段,不是一般的~near~或~far~跳转指令能解决得了的,所以这里我们需要一个特别的跳转指令。在这条指令运行之前,所有的指令都是~16~位的,在它运行之后,就变成~32~位指令的世界。

在~Intel~的手册中,把这条混合跳转指令称为~far jump(ptr16:32)~,在~NASM~手册中,将这条指令称为~Mixed-Size Jump~,我们就沿用~NASM~的说法,将这条指令称为混合字长跳转指令。NASM~提供了这条指令的汇编语言实现:
\begin{Command}
jmp dword 0x1234:0x56789ABC
\end{Command}
NASM~的手册中说~GAS~没有提供这条指令的实现,我就用~.byte~伪代码直接写了二进制指令:
\begin{Command}
/* Mixed-Size Jump. */
.2byte  0xea66
.4byte  0x00000000
.2byte  SelectorCode32
\end{Command}
但是有位朋友提醒我说现在的~GAS~已经支持混合字长跳转指令(如图~\ref{CHpm_mixedjmp}),看来~NASM~的手册好久没有维护喽~\smiley~。

\VerbatimInput[fontfamily=tt,fontsize=\footnotesize,frame=lines, framerule=0.4mm, numbers=left, numbersep=3pt, tabsize=2, firstline=77, lastline=82]{../src/chapter3/1/loader.S}
\codecaption{混合字长跳转指令(节自 chapter3/1/loader.S)}\label{CHpm_mixedjmp}

执行这条混合字长的跳转指令时,CPU~就会用段选择子~\code{SelectorCode32}~去寻找~GDT~中对应的段,由于段偏移是~0~,所以~CPU~将跳转到图~\ref{CHpm_demo1}~中~demo~程序的开头。为了方便阅读,整个~loader.S~的代码附在图~\ref{CHpm_loader1}~中:

\VerbatimInput[fontfamily=tt,fontsize=\footnotesize,frame=lines, framerule=0.4mm, numbers=left, numbersep=3pt, tabsize=2, firstline=1, lastline=99]{../src/chapter3/1/loader.S}
\codecaption{chapter3/1/loader.S (节自 chapter3/1/loader.S)}\label{CHpm_loader1}

\subsection{生成镜像并测试}

使用与第~\ref{CHsmall_test}~节完全相同的方法,我们可以将代码编译并将~LOADER.BIN~拷贝到镜像文件中。利用最新的镜像文件启动~VirtualBox~我们得到图~\ref{vb_run_5}~。

可以看到,屏幕的左侧中央打出了一个红色的~P~,这就是我们那个在保护模式下运行的简单~demo~所做的事情,这说明我们的代码是正确的。从实模式迈入保护模式,这只是一小步,但对于我们的操作系统来说,这是一大步。从此我们不必再被限制到~20~位的地址空间中,有了更大的自由度。

\FIG{第一次进入保护模式}{vb_run_5}{0.75\textwidth}

\section{段式存储}

如果您仔细阅读了图~\ref{protected_seg}~,您就会发现图中并未提到~GDT~,而是使用的~Descriptor Table(DT)~。这是因为对于~x86~架构的~CPU~来说,~DT~总共有两个:我们上节介绍过的~GDT~和下面要介绍的~LDT~。这两个描述符表构成了~x86 CPU~段式存储的基础。顾名思义,~GDT~作为全局的描述符表,只能有一个,而~LDT~作为局部描述符表,就可以有很多个,这也是以后操作系统给每个任务分配自己的存储空间的基础。

\subsection{LDT~数据结构} \label{CHpm_ldt}

\FIG{段选择子数据结构}{seg_selector}{0.6\textwidth}

事实上,~LDT~和~GDT~的差别非常小,~LDT~段描述符的数据结构和图~\ref{seg_desc}~所示是一样的。所不同的就是,~LDT~用指令~\code{lldt}~来加载,并且指向~LDT~描述符项的段选择子的~TI~位置必须标识为~1~,如图~\ref{seg_selector}~所示。这样,在使用~TI flag := 1~的段选择子时,操作系统才会从当前的~LDT~而不是~GDT~中去寻找对应的段描述符。

这里值得注意的一点是:GDT~是由线性空间里的地址定义的,即~\code{lgdt}~指令的参数是一个线性空间的地址;而~LDT~是由~GDT~中的一个段描述符定义的,即~\code{lldt}~指令的参数是~GDT~中的一个段选择子。这是因为在加载~GDT~之前寻址模式是实模式的,而加载~GDT~后寻址模式变成保护模式寻址,将~LDT~作为~GDT~中的段使用,也方便操作系统在多个~LDT~之间切换。

\subsection{段描述符属性} \label{CHpm_desattr}

我们在介绍图~\ref{seg_desc}~时,并没有完全介绍段描述符的各个~Flag~和可能的属性,这一小节就用来专门介绍段描述符的属性,按照图~\ref{seg_desc}~中的~Flag~从左向右的顺序:

\begin{itemize}
\item{\textbf{G}}: G(Granularity,粒度):如果~G flag~置为~0~,段的大小以字节为单位,段长度范围是~\code{1 byte$\sim$1 MB}~;如果~G flag~置为~1~,段的大小以~4 KB~为单位,段长度范围是~\code{4 KB $\sim$ 4 GB}~。
\item{\textbf{D/B}}:D/B(Default operation size/Default stack pionter size and/or upper Bound, 默认操作大小),其意思取决于段描述符是代码段、数据段或者堆栈段。该~flag~置为~0~代表代码段/数据段为~16~位的;置为~1~代表该段是~32~位的。
\item{\textbf{L}}:L(Long, 长),~L flag~是~IA-32e(Extended Memory 64 Technology)~模式下使用的标志。该~flag~置为~1~代表该段是正常的~64~位的代码段;置为~0~代表在兼容模式下运行的代码段。在~IA-32~架构下,该位是保留位,并且永远被置为~0~。
\item{\textbf{AVL}}:保留给操作系统软件使用的位。
\item{\textbf{P}}:P(segment-Present,段占用?) flag~用于标志段是否在内存中,主要供内存管理软件使用。如果~P flag~被置为~0~,说明该段目前不在内存中,该段指向的内存可以暂时被其它任务占用;如果~P flag~被置为~1~,说明该段在内存中。如果~P flag~为~0~的段被访问,处理机会产生一个~segment-not-present(\#NP)~异常。
\item{\textbf{DPL}}:DPL(Descriptor Privilege Level)域标志着段的优先级,取值范围是从~0$\sim$3(2-bit)~,0~代表着最高的优先级。关于优先级的作用,我们将在下节讨论。
\item{\textbf{S}}:S(descriptor type) flag~标志着该段是否系统段:置为~0~代表该段是系统段;置为~1~代表该段是代码段或者数据段。
\item{\textbf{Type}}:Type~域是段描述符里最复杂的一个域,而且它的意义对于代码/数据段描述符和系统段/门描述符是不同的,下面我们用两张表来展示当~Type~置为不同值时的意义。

表~\ref{code_data_types}~所示即为代码/数据段描述符的所有~Type~可能的值(0-15, 4-bit)以及对应的属性含意,表~\ref{sys_gate_types}~所示为系统段/门描述符的~Type~可能的值以及对应的属性含意。这两张表每个条目的内容是自明的,而且我们在后面的讨论中将不止一次会引用这两张表的内容,所以这里对每个条目暂时不加详细阐述。

\begin{center}\begin{longtable}{c|c|c|c|c|c|l}
\caption[]{代码/数据段描述符的~Type~属性列表}\label{code_data_types}\\
\hline
\multicolumn{5}{c|}{\textbf{Type Field}} & \textbf{Descriptor Type} & \textbf{Description}\bigstrut\\
\cline{1-5}
\textbf{Decimal} & \textbf{11} & \textbf{10} & \textbf{9} & \textbf{8} & & \\
        &    &  \textbf{E} & \textbf{W} & \textbf{A} & & \\
\hline
0 & 0 & 0 & 0 & 0 & Data & Read-Only\\
1 & 0 & 0 & 0 & 1 & Data & Read-Only, Accessed\\
2 & 0 & 0 & 1 & 0 & Data & Read/Write\\
3 & 0 & 0 & 1 & 1 & Data & Read/Write, Accessed\\
4 & 0 & 1 & 0 & 0 & Data & Read-Only, Expand-down\\
5 & 0 & 1 & 0 & 1 & Data & Read-Only, Expand-down, Accessed\\
6 & 0 & 1 & 1 & 0 & Data & Read/Write, Expand-down\\
7 & 0 & 1 & 1 & 1 & Data & Read/Write, Expand-down, Accessed\\
\hline
        &    &  \textbf{C} & \textbf{R} & \textbf{A} & & \\
\hline
8 & 1 & 0 & 0 & 0 & Code & Execute-Only\\
9 & 1 & 0 & 0 & 1 & Code & Execute-Only, Accessed\\
10 & 1 & 0 & 1 & 0 & Code & Execute/Read\\
11 & 1 & 0 & 1 & 1 & Code & Execute/Read, Accessed\\
12 & 1 & 1 & 0 & 0 & Code & Execute-Only, Conforming\\
13 & 1 & 1 & 0 & 1 & Code & Execute-Only, Conforming, Accessed\\
14 & 1 & 1 & 1 & 0 & Code & Execute/Read-Only, Conforming\\
15 & 1 & 1 & 1 & 1 & Code & Execute/Read-Only, Conforming, Accessed\\
\hline
\end{longtable}\end{center}

\begin{center}\begin{longtable}{c|c|c|c|c|l}
\caption[]{系统段/门描述符的~Type~属性列表}\label{sys_gate_types}\\
\hline
\multicolumn{5}{c|}{\textbf{Type Field}} & \textbf{Description}\bigstrut\\
\hline
\textbf{Decimal} & \textbf{11} & \textbf{10} & \textbf{9} & \textbf{8} & \textbf{32-Bit Mode}\\
\hline
0 & 0 & 0 & 0 & 0 & Reserved\\
1 & 0 & 0 & 0 & 1 & 16-bit TSS(Available)\\
2 & 0 & 0 & 1 & 0 & LDT\\
3 & 0 & 0 & 1 & 1 & 16-bit TSS(Busy)\\
4 & 0 & 1 & 0 & 0 & 16-bit Call Gate\\
5 & 0 & 1 & 0 & 1 & Task Gate\\
6 & 0 & 1 & 1 & 0 & 16-bit Interrupt Gate\\
7 & 0 & 1 & 1 & 1 & 16-bit Trap Gate\\
8 & 1 & 0 & 0 & 0 & Reserved\\
9 & 1 & 0 & 0 & 1 & 32-bit TSS(Available)\\
10 & 1 & 0 & 1 & 0 & Reserved\\
11 & 1 & 0 & 1 & 1 & 32-bit TSS(Busy)\\
12 & 1 & 1 & 0 & 0 & 32-bit Call Gate\\
13 & 1 & 1 & 0 & 1 & Reserved\\
14 & 1 & 1 & 1 & 0 & 32-bit Interrupt Gate\\
15 & 1 & 1 & 1 & 1 & 32-bit Trap Gate\\
\hline
\end{longtable}\end{center}

\end{itemize}

\subsection{使用~LDT~}

从目前的需求来看,对~LDT~并没有非介绍不可的理由,但是理解~LDT~的使用,对理解段式存储和处理机多任务存储空间分配有很大的帮助。所以我们在下面的代码中实现几个简单的例子:一,建立~32~位数据和堆栈两个段并将描述符添加到~GDT~中;二,添加一段简单代码,并以其段描述符为基础建立一个~LDT;三,在~GDT~中添加~LDT~的段描述符并初始化所有~DT~;四,进入保护模式下运行的~32~位代码段后,加载~LDT~并跳转执行~LDT~中包含的代码段。

首先,建立~32~位全局数据段和堆栈段,并将其描述符添加到~GDT~中:

\VerbatimInput[fontfamily=tt,fontsize=\footnotesize,frame=topline, framerule=0.4mm, numbers=left, numbersep=3pt, tabsize=2, firstline=50, lastline=62]{../src/chapter3/2/loader.S}
\VerbatimInput[fontfamily=tt,fontsize=\footnotesize,frame=bottomline, framerule=0.4mm, numbers=left, numbersep=3pt, tabsize=2, firstline=22, lastline=41]{../src/chapter3/2/loader.S}
\codecaption{32~位全局数据段和堆栈段,以及对应的~GDT~结构(节自 chapter3/2/loader.S)}\label{CHpm_add_ds}

在图~\ref{CHpm_add_ds}~中,我们首先建立了一个全局的数据段,并在数据段里放置了两个字符串,分别用来进入保护模式后和跳转到~LDT~指向的代码段后作为信息输出。然后又建立了一个全局的堆栈段,为堆栈段预留了~512~字节的空间,并将栈顶设置为距栈底~511~字节处。然后与上节介绍的类似,将数据段和堆栈段的段描述符添加到~GDT~中,并设置好对应的段选择子。

要注意到数据段、堆栈段和代码段的段描述符属性不尽相同。数据段的段描述符属性是~\code{DA\_DRW}~,回忆我们前面~pm.h~的内容(图~\ref{CHpm_segattr}~),~\code{DA\_DRW}~的内容是~\code{0x92},用二进制就是~\code{10010010},其后四位就对应着图~\ref{code_data_types}~中的第二~2(0010)~项,说明这个段是可读写的数据段;前四位对应着~\code{P|DPL|S}~三个~flag~,即~\code{P:1, DPL:00, S:1}~,与第~\ref{CHpm_desattr}~节结合理解,意思就是该段在内存中,为最高的优先级,非系统段。所以我们可以看到~pm.h~中的各个属性变量定义,就是将二进制的属性值用可理解的变量名表示出来,在用的时候直接加上变量即可。

同理我们也可以分别来理解~GDT~中堆栈段和代码段描述符的属性定义。因为不同类型的属性使用的是段描述符中不同的位,所以不同类型的属性可以直接相加得到复合的属性值,例如堆栈段的~\code{(DA\_DRWA + DA\_32)}~,其意思类似于~\code{C++}~中~\code{fstream}~打开文件时可以对模式进行算术或(\code{ios\_base::in | ios\_base::out})来得到复合参数。

其次,添加一段简单的代码,并以其描述符为基础建立一个~LDT:

\VerbatimInput[fontfamily=tt,fontsize=\footnotesize,frame=topline, framerule=0.4mm, numbers=left, numbersep=3pt, tabsize=2, firstline=114, lastline=140]{../src/chapter3/2/loader.S}
\VerbatimInput[fontfamily=tt,fontsize=\footnotesize,frame=bottomline, framerule=0.4mm, numbers=left, numbersep=3pt, tabsize=2, firstline=42, lastline=49]{../src/chapter3/2/loader.S}
\codecaption{32~位代码段,以及对应的~LDT~结构(节自 chapter3/2/loader.S)}\label{CHpm_build_ldt}

\code{LABEL\_CODEA}~就是我们为~LDT~建立的简单代码段,其作用就是操作显存在屏幕的第~12~行开始用红色的字打印出偏移~\code{OffsetLDTMessage}~指向的全局数据段中的字符串。下面就是以~\code{LABEL\_CODEA}~为基础建立的~LDT~,从~LDT~的结构来说,与~GDT~没有区别,但是我们不用像~\code{GdtPtr}~再建立一个~\code{LdtPtr}~,因为~LDT~实际上是在~GDT~中定义的一个段,不用实模式的线性地址表示。

LDT~的选择子是与~GDT~选择子有明显区别的,图~\ref{seg_selector}~清楚地解释了这一点,所以指向~LDT~的选择子都应该将~TI~位置~1~,在图~\ref{CHpm_build_ldt}~的最后一行也实现了这一操作。

第三,在~GDT~中添加~LDT~的段描述符(在图~\ref{CHpm_add_ds}~中我们已经能看到在~GDT~中添加好了~LDT~的段描述符),初始化所有段描述符。由于初始化段描述符属于重复性工作,我们在~pm.h~中添加一个汇编宏~\code{InitDesc}~来帮我们做这件事情。

\VerbatimInput[fontfamily=tt,fontsize=\footnotesize,frame=lines, framerule=0.4mm, numbers=left, numbersep=3pt, tabsize=2, firstline=84, lastline=96]{../src/chapter3/2/pm.h}
\codecaption{自动初始化段描述符的宏代码(节自 chapter3/2/pm.h)}\label{CHpm_initdesc}

\VerbatimInput[fontfamily=tt,fontsize=\footnotesize,frame=lines, framerule=0.4mm, numbers=left, numbersep=3pt, tabsize=2, firstline=63, lastline=112]{../src/chapter3/2/loader.S}
\codecaption{在实模式代码段中初始化所有段描述符(节自 chapter3/2/loader.S)}\label{CHpm_init_dts}

初始化各个段描述符的方式与上一节介绍的初始化~GDT~描述符的方式没有什么本质不同,因为属性都已经预设好,运行时只需要将段地址填入描述符中的地址域即可,代码都是重复的。我们引入宏~\code{InitDesc}的帮助,能大大缩短代码长度,增强代码的可读性。

第四,进入保护模式下运行的~32~位代码段后,加载~LDT~并跳转执行~LDT~中包含的代码段:

\VerbatimInput[fontfamily=tt,fontsize=\footnotesize,frame=lines, framerule=0.4mm, numbers=left, numbersep=3pt, tabsize=2, firstline=141, lastline=176]{../src/chapter3/2/loader.S}
\codecaption{在保护模式代码段中加载~LDT~并跳转执行~LDT~代码段(节自 chapter3/2/loader.S)}\label{CHpm_run_ldt}

在~\code{LABEL\_SEG\_CODE32}~中前几行,我们可以看到非常熟悉的汇编指令,和一般汇编程序开头初始化数据/代码/堆栈段寄存器的指令非常像,只不过这里赋给几个寄存器的参数都是段选择子,而不是一般的地址。该代码段剩下的内容和前面图~\ref{CHpm_build_ldt}~中~\code{LABEL\_CODEA}~一样,都是打印一个字符串,只不过这里选择在第~10~行(屏幕左侧中央)打印。

为了方便阅读,整个~loader.S~的代码附在图~\ref{CHpm_loader2}~中。

\VerbatimInput[fontfamily=tt,fontsize=\footnotesize,frame=lines, framerule=0.4mm, numbers=left, numbersep=3pt, tabsize=2]{../src/chapter3/2/loader.S}
\codecaption{chapter3/2/loader.S}\label{CHpm_loader2}

\subsection{生成镜像并测试}

使用与第~\ref{CHsmall_test}~节完全相同的方法,我们可以将代码编译并将~LOADER.BIN~拷贝到镜像文件中。利用最新的镜像文件启动~VirtualBox~我们得到图~\ref{vb_run_6}~。

\FIG{第一次进入保护模式}{vb_run_6}{0.75\textwidth}

可以看到,该程序首先在屏幕左侧中央(第~10~行)打印出来~\code{"Welcome to protect mode!\^{}-\^{}"}~,这是由~GDT~中的~32~位代码段~\code{LABEL\_SEG\_CODE32}~打印出来的,标志着我们成功进入保护模式;然后在屏幕的第~12~行打印出来~\code{"Aha, you jumped into a LDT segment."}~,这个是由~LDT~中的~32~位代码段~\code{LABEL\_CODEA}~打印出来的,标志着~LDT~的使用正确。因为这两个字符串都是被存储在~32~位全局数据段中,这两个字符串的成功打印也说明在~GDT~中添加的数据段使用正确。

\subsection{段式存储总结}

段式存储和页式存储都是最流行的计算机内存保护方式。段式存储的含义简单来说就是先将内存分为各个段,然后再分配给程序供不同用途使用,并保证对各个段的访问互不干扰。x86~主要使用段寄存器(得到的段基址)~\code{+}~偏移量来访问段中数据,也简化了寻址过程。

在~x86~的初期实模式下就使用着非常简单的段式存储方式,如图~\ref{rm_addr}~所示,这种模式下分段主要是为了方便寻址和隔离,没有提供其它的保护机制。x86~保护模式下采用了更高级的段式存储方式:用全局和局部描述符表存储段描述符信息,使用段选择子表示各个段描述符,如图~\ref{protected_seg}~所示。

由于保护模式使用段描述符来保存段信息而不是像实模式一样直接使用段地址,在段描述符中就可以添加一些属性来限制对段的访问权限,如我们在第~\ref{CHpm_desattr}~节中讨论的那样。这样,通过在访问段时检查权限和属性,就能做到对程序段的更完善保护和更好的内存管理。

x86~使用全局描述符表(GDT)和局部描述符表(LDT)来实现不同需求下对程序段的控制,操作系统使用唯一的一个~GDT~来维护一些和系统密切相关的段描述符信息,为不同的任务使用不同的~LDT~来实现对多任务内存管理的支持,简化了任务切换引起的内存切换的难度。

\section{特权级}

\BOXED{0.9\textwidth}{
\danger\\ 如果您需要更详细的知识,也许您更愿意去读 Intel 的手册,本节内容主要集中在:\href{http://download.intel.com/design/processor/manuals/253668.pdf}{Intel\textregistered~64 and IA-32 Architectures Software Developer's Manual, Volume 3A: System Programming Guide}, 第~4~章.\enddanger
}

特权级是为了保护处理机资源而引入的概念。将同一个处理机上执行的不同任务赋予不同的特权级,可以控制该任务可以访问的资源,比如内存地址范围、输入输出端口、和一些特殊指令的使用。在~x86~体系结构中,共有~4~个特权级别,0~代表最高特权级,3~代表最低特权级。由于在~x86~体系结构中,n~级可以访问的资源均可以被~0~到~n~级访问,这个模式被称作~ring~模式,相应地我们也将~x86~的对应特权级称作~ring n。

现代的~PC~操作系统的内核一般工作在~ring 0~下,拥有最高的特权级,应用程序一般工作在~ring 3~下,拥有最低的优先级。虽然~x86~体系结构提供了~4~个特权级,但操作系统并不需要全部使用到这~4~个级别,可以根据需要来选择使用几个特权级。比如~Linux/Unix~和~Windows NT~,都是只使用了~0~级和~3~级,分别用于内核模式和用户模式;而~DOS~则只使用了~0~级。

为了实施对代码段和数据段的特权级检验,x86~处理机引入了以下三种特权级类型(请注意这里提到的特权级高低均为实际高低,而非数值意义上的高低):

\begin{itemize}
\item{\textbf{CPL(Current Privilege Level)}}:当前特权级,存储在~CS~和~SS~的~0, 1~位。它代表当前执行程序或任务的特权级,通常情况下与当前执行指令所在代码段的~DPL~相同。当程序跳转到不同特权级的代码段时,CPL~会随之修改。当访问一致代码段(Conforming Code Segment)时,对~CPL~的处理有些不同。一致代码段可以被不高于(数值上大于等于)该段~DPL~的特权级代码访问,但是,CPL~在访问一致代码段时不会跟随~DPL~的变化而更改。
\item{\textbf{DPL(Descriptor Privilege Level)}}:描述符特权级,定义于段描述符或门描述符中的~DPL~域(见图~\ref{seg_desc}),它限制了可以访问此段资源的特权级别。根据被访问的段或者门的不同,DPL~的意义也不同:
  \begin{itemize}
  \item{\textbf{数据段}}:数据段的~DPL~限制了可以访问该数据段的最低特权级。假如数据段的~DPL~为~1,那么只有~CPL~为~0,1~的程序才能访问该数据段。
  \item{\textbf{非一致代码段(不使用调用门)}}:非一致代码段就是一般的代码段,它的~DPL~表示可以访问该段的特权级,程序或者任务的特权级必须与该段的~DPL~完全相同才可以访问该段。
  \item{\textbf{调用门}}:调用门的~DPL~限制了可以访问该门的最低特权级,与数据段~DPL~的意思一样。
  \item{\textbf{一致代码段和使用调用门访问的非一致代码段}}:这种代码段的~DPL~表示可以访问该段的最高特权级。假如一致代码段的~DPL~是~2,那么~CPL~为~0,1~的程序就无法访问该段。
  \item{\textbf{TSS(Task State Segment)}}:任务状态段的~DPL~表示可以访问该段的最低特权级,与数据段~DPL~的意思一样。
  \end{itemize}
比如一个数据段的~DPL~为~1~,那么只有当前特权级(CPL)为~0~或~1~的程序可以访问该段。
\item{\textbf{RPL(Requested Privilege Level)}}:请求特权级,定义于段选择子的~RPL~域中(见图~\ref{seg_selector})。它限制了这个选择子可访问资源的最高特权级。比如一个段选择子的~RPL~为~3~,那么使用这个段选择子只能访问~DPL~为~3~或者~4~的段,即使使用这个段选择子的程序当前特权级(CPL)为~0~。就是说,$\max{(CPL, RPL)}\le DPL$~才被允许访问该段,即当~CPL~小于~RPL~时,RPL~起决定性作用,反之亦然。使用~RPL~可以避免特权级高的程序代替应用程序访问该应用程序无权访问的段。比如在系统调用时,应用程序调用系统过程,虽然系统过程的优先级高($CPL=0$),但是被调用的系统过程仍然无法访问特权级高于应用程序的段($DPL<RPL=4$),就避免了可能出现的安全问题。
\end{itemize}

在将段描述符对应的段选择子加载到段寄存器时,处理机通过将~CPL, 段选择子的~RPL~和该段的~DPL~相比较,来判断程序是否有权访问另外一个段。如果~$CPL>\max{(RPL, DPL)}~$,或者$\max{(CPL, RPL)}>DPL$,那么该访问就是不合法的,处理机就会产生一个常规保护异常(\texttt{\#}GP, General Protection Exception)。

\subsection{不合法的访问请求示例}

我们来看一个不合法的访问请求的例子,在上一节的~loader.S~中把~\code{LABEL\_DESC\_DATA}~对应的描述符的~\code{DPL}~设置为~1,然后将该数据段对应的段选择子的~RPL~设置为~3,即修改以下两行:
\begin{Command}
LABEL_DESC_DATA:    Descriptor        0,      (DataLen - 1), (DA_DRW + DA_DPL1)
.set    SelectorData,   (LABEL_DESC_DATA   - LABEL_GDT + SA_RPL3)
\end{Command}

\FIGFIX{虚拟机出现异常,黑屏}{vb_run_7}{0.75\textwidth}

\FIGFIX{虚拟退出后~VBox~主窗口显示~Abort}{vb_main_4}{0.75\textwidth}

再~\code{make, sudo make copy},用~VirtualBox~加载生成的镜像运行一下,就会发现虚拟机黑屏一会儿就会退出(如图~\ref{vb_run_7}),然后~VirtualBox~主窗口中显示该虚拟机~Aborted(如图~\ref{vb_main_4})。这是因为我们违反特权级的规则,使用~RPL=3~的选择子去访问~DPL=1~的段,这个不合法的访问请求引起处理机产生常规保护异常(\texttt{\#}GP)。而我们又没有准备对应的异常处理模块,当处理机找不到异常处理程序时就只好退出了。


\subsection{控制权转移的特权级检查}

在将控制权从一个代码段转移到另一个代码段之前,目标代码段的段选择子必须被加载到~CS~中。处理器在这个过程中会查看目标代码段的段描述符以及对其界限、类型(见图~\ref{seg_desc})和特权级进行检查。如果没有错误发生,CS~寄存器会被加载,程序控制权被转移到新的代码段,从~EIP~指示的位置开始运行。

JMP, CALL, RET, SYSENTER, SYSEXIT, INT n~和~IRET~这些指令,以及中断和异常机制都会引起程序控制权的转移。

JMP~和~CALL~指令可以实现以下~4~种形式的转移:

\begin{itemize}
\item 目标操作数包含目标代码段的段选择子。
\item 目标操作数指向一个包含目标代码段段选择子的调用门描述符。
\item 目标操作数指向一个包含目标代码段段选择子的任务状态段。
\item 目标操作数指向一个任务门,这个任务门指向一个包含目标代码段段选择子的任务状态段。
\end{itemize}

下面两个小节将描述前两种转移的实现方法,后两种控制权转移方法我们将在用到时再进行解释。

\subsubsection{用~JMP~或~CALL~直接转移}

用~JMP, CALL~和~RET~指令在段内进行近跳转并没有特权级的变化,所以对这类转移是不进行特权级检查的;用~JMP, CALL~和~RET~在段间进行远跳转涉及到其它代码段,所以要进行特权级检查。

对不通过调用门的直接转移来说,又分为两种情形:

\begin{itemize}
\item{访问非一致代码段}:当目标是非一致代码段时(目标段段描述符的~C flag~为~0~,见图~\ref{code_data_types}),特权级检查要求调用者的~CPL~与目标代码段的~DPL~相等,而且调用者使用的目标代码段段选择子的~RPL~必须小于等于目标代码段的~DPL。我们之前的代码都属于这种情形,其中~$CPL=DPL=RPL=0$。
\item{访问一致代码段}:当目标是一致代码段时(目标段段描述符的~C flag~为~1~,见图~\ref{code_data_types}),特权级检查要求~$CPL \ge DPL$,RPL~不被检查,而且转移时并不修改~CPL。
\end{itemize}

总的来说,通过~JMP~和~CALL~实行的都是一般的转移,最多从低特权级转移到高特权级的一致代码段,CPL~总是不变的。

\subsection{使用调用门转移}

调用门是~x86~体系结构下用来控制程序在不同特权级间转移的一种机制。它的目的是使低特权级的代码能够调用高特权级的代码,这一机制在使用了内存保护和特权级机制的现代操作系统中非常有用,因为它允许应用程序在操作系统控制下安全地调用内核例程或者系统接口。

\FIG{调用门描述符}{callgate_desc}{.9\textwidth}

门其实也是一种描述符,和段描述符类似。调用门描述符的数据结构如图~\ref{callgate_desc}~所示。其实看起来这个调用门描述符的数据结构要比段描述符简单一些,至少从它的属性来说,没有段描述符多。我们仍然只关注最重要的部分:首先是段选择子(Segment Selector),指定了通过这个调用门访问的代码段;其次是段偏移量(Offset in Segment),指定了要访问代码段中的某个入口偏移;描述符特权级(DPL),代表此门描述符的特权级;P,代表此调用门是否可用;参数计数(Param. Count)记录了如果发生栈切换的话,有多少个选项参数会在栈间拷贝。

简单来说,调用门描述了由一个段选择子和一个偏移所指定的目标代码段中的一个地址,程序通过调用门将转移到这个地址。下面我们通过一个简单的例子来介绍一下调用门的基本使用方法。

\subsubsection{简单的调用门转移举例}

为了使用调用门,我们首先要给出一个目标段,然后用该目标段的信息初始化调用门的门描述符,最后用调用门的门选择子实现门调用。

添加一个目标段我们已经做过很多次,非常简单。首先在上一节~loader.S~最后添加一个打印一个字符的代码段~\code{LABEL\_SEG\_CODECG},接着将该段的段描述符~\code{LABEL\_DESC\_CODECG}~添加到~GDT~中,然后为该段准备一个段选择子~\code{SelectorCodeCG},最后加入初始化该段描述符的代码:

\VerbatimInput[fontfamily=tt,fontsize=\footnotesize,frame=topline, framerule=0.4mm, numbers=left, numbersep=3pt, tabsize=2, firstline=197, lastline=211]{../src/chapter3/3/loader.S}
\VerbatimInput[fontfamily=tt,fontsize=\footnotesize,frame=none, framerule=0.4mm, numbers=left, numbersep=3pt, tabsize=2, firstline=28, lastline=29]{../src/chapter3/3/loader.S}
\VerbatimInput[fontfamily=tt,fontsize=\footnotesize,frame=none, framerule=0.4mm, numbers=left, numbersep=3pt, tabsize=2, firstline=43, lastline=44]{../src/chapter3/3/loader.S}
\VerbatimInput[fontfamily=tt,fontsize=\footnotesize,frame=bottomline, framerule=0.4mm, numbers=left, numbersep=3pt, tabsize=2, firstline=92, lastline=94]{../src/chapter3/3/loader.S}
\codecaption{添加调用门的目标段(节自 chapter3/3/loader.S)}\label{CHpm_codecg}

总的来看,~\code{LABEL\_SEG\_CODECG}~指向的这个段和我们以前为了打印程序运行结果所使用的段没有本质不同,为了简单起见,这里我们仅仅让它打印一个字符~\code{'C'}~就返回。

用目标代码段~\code{LABEL\_SEG\_CODECG}~的信息初始化调用门的门描述符~\code{LABEL\_CG\_TEST},以及门选择子~\code{SelectorCGTest}。与汇编宏~\code{Descriptor}~类似,我们这里使用汇编宏~\code{Gate}~来初始化门描述符,宏~\code{Gate}~的定义可以在头文件~\code{pm.h}~中找到:

\VerbatimInput[fontfamily=tt,fontsize=\footnotesize,frame=lines, framerule=0.4mm, numbers=left, numbersep=3pt, tabsize=2, firstline=71, lastline=82]{../src/chapter3/3/pm.h}
\codecaption{汇编宏~Gate~定义(节自 chapter3/3/pm.h)}\label{CHpm_cg_gate}

\VerbatimInput[fontfamily=tt,fontsize=\footnotesize,frame=topline, framerule=0.4mm, numbers=left, numbersep=3pt, tabsize=2, firstline=29, lastline=33]{../src/chapter3/3/loader.S}
\VerbatimInput[fontfamily=tt,fontsize=\footnotesize,frame=bottomline, framerule=0.4mm, numbers=left, numbersep=3pt, tabsize=2, firstline=43, lastline=45]{../src/chapter3/3/loader.S}
\codecaption{设置调用门描述符及选择子(节自 chapter3/3/loader.S)}\label{CHpm_cg_test}

我们可以看到,宏~\code{Gate}~的四个参数分别为:段选择子、偏移量、参数计数和属性,它们在存储空间中的分布与图~\ref{callgate_desc}~中介绍相同。由于这个例子仅仅介绍调用门的简单使用,并不涉及特权级切换,所以也不发生栈切换,这里我们将参数计数设置为~0;门描述符的属性为~\code{(DA\_386CGate + DPL)}~,表明它是一个调用门(属性定义见图~\ref{CHpm_descm}),DPL~为~0~,与我们一直使用的特权级相同;目标代码段选择子是~\code{SelectorCodeCG}~,偏移是~0~,所以如果该调用门被调用,将转移到目标代码段的开头,即~\code{LABEL\_SEG\_CODECG}~处开始执行。

使用远调用~lcall~指令调用该调用门的门选择子~\code{SelectorCGTest}:

\VerbatimInput[fontfamily=tt,fontsize=\footnotesize,frame=lines, framerule=0.4mm, numbers=left, numbersep=3pt, tabsize=2, firstline=185, lastline=190]{../src/chapter3/3/loader.S}
\codecaption{调用门选择子(节自 chapter3/3/loader.S)}\label{CHpm_cg_call1}

由于对调用门的调用往往涉及到段间转移,所以我们通常使用~gas~的~lcall~远跳转指令和~lret~远返回指令进行调用和返回。

这样我们就完成了使用调用门进行简单控制权转移的代码,~\code{make, sudo make copy}~之后,用~VBox~虚拟机载入生成的镜像,运行结果如图~\ref{vb_run_8}~所示。由于我们仅仅是在加载~LDT~之前添加了一个门调用,而且门调用的目标段在屏幕的第~11~行第~0~列打印了一个~\code{'C'}~后就返回到了调用处,所以加载~LDT~的代码继续运行,就是图中所示结果。

\FIG{使用调用门进行简单的控制权转移}{vb_run_8}{0.75\textwidth}

\subsubsection{涉及特权级变化的调用门转移}

在上面例子中我们只是使用调用门取代了传统的直接跳转方式,并没有涉及到特权级的变化。显然调用门不是用来做这种自找麻烦的事情的,其设计的主要目的是实现从低特权级代码跳转到高特权级的非一致代码的功能。

在使用调用门进行转移时,处理机会使用四个特权级值来检查控制权的转移是否合法:

\begin{enumerate}
\item CPL:当前特权级;
\item RPL:调用门选择子的请求特权级;
\item DPL:调用门描述符的描述符特权级;
\item DPL:目标代码段的段描述符特权级。
\end{enumerate}

使用~CALL~或者~JMP~指令访问调用门进行控制权转移时,特权级检查的规则有所不同,如表~\ref{callgate_rules}~所示:

\begin{center}\begin{longtable}{|l|l|}
\caption[]{调用门特权级检查规则}\label{callgate_rules}\\
\hline
\textbf{指令} & \textbf{特权级检查规则}\\
\hline
CALL & CPL $\le$ 调用门~DPL; RPL $\le$ 调用门~DPL\\
     & 目标段 DPL $\le$ CPL\\
\hline
JMP  & CPL $\le$ 调用门~DPL; RPL $\le$ 调用门~DPL\\
     & 当目标段是一致代码段时:目标段~DPL $\le$ CPL\\
     & 当目标段是非一致代码段时:目标段~DPL = CPL\\
\hline
\end{longtable}\end{center}

这张表内容虽然不多,但也不容易很快理解。这里我们应该着重看目标段~DPL~和~CPL~的比较,这才是调用门特权级检验的特点所在。总的来说,使用调用门需要目标段的~DPL~小于或等于~CPL~,意思就是要转移的目标段特权级高于当前特权级。这与我们在本节开头看到的一般转移的特权级检查有非常明显的不同。除此之外剩下的检查就是对调用门访问的检查了,这种特权级检查和访问一个数据段时进行的特权级检查的规则是一样的,我们已经熟知了。

我们已经了解了涉及特权级变化的调用门转移时处理机进行的特权级检查规则,但为了写一段从低特权级转移到高特权级的测试代码,我们仍然需要处理一个问题:如何从高特权级转移到低特权级?因为我们之前的代码一直运行在~ring 0~特权级上,要实现从低到高的转移,首先要从高特权级转到低特权级。

其实思想很简单,既然~CALL~指令能从低特权级转移到高特权级,自然而然地~RET~指令能从高特权级返回低特权级。我们只需要~hack~一下~RET~指令的使用方法即可(即用~RET~指令实现跳转到低特权级代码的功能)。

一般~CALL~和~RET~指令都是配合使用的,先用~CALL~跳转到目标地址,目标代码执行完后再用~RET~返回到~CALL~的下一条指令。为了从~ring 0~跳转到~ring 3~,我们并不使用~CALL~而直接执行~RET。为了使~RET~执行返回时不出错,我们需要为~RET~准备好返回时环境,就像通常执行~CALL~指令后处理机进行的工作一样。

那么执行~CALL~指令时处理机都进行了哪些工作呢?这是一个非常复杂的问题,我们将留待下个小节再详细介绍。但是在我们的例子里(即最简单的情况下),CALL~所做的就是将~SS, ESP, CS, EIP~这四个寄存器的值顺序压到栈里,这样在~RET~指令执行的时候,处理机从堆栈~pop~出~EIP, CS, ESP, SS~的值来恢复~CALL~指令执行后的处理机现场。所以为了使~RET~跳转到我们想要执行的代码段,我们只需要手动将~ring 3~目标代码段的对应的~SS, ESP, CS, EIP~值压到栈里即可。

下面开始准备这个~demo,仍旧是在上一节代码的基础上进行添加。首先,我们准备一个~ring 3~目标代码段和新栈:

\VerbatimInput[fontfamily=tt,fontsize=\footnotesize,frame=lines, framerule=0.4mm, numbers=left, numbersep=3pt, tabsize=2, firstline=224, lastline=237]{../src/chapter3/4/loader.S}
\codecaption{要运行在~ring 3~下的代码段(节自 chapter3/4/loader.S)}\label{CHpm_coder3}

\VerbatimInput[fontfamily=tt,fontsize=\footnotesize,frame=lines, framerule=0.4mm, numbers=left, numbersep=3pt, tabsize=2, firstline=73, lastline=76]{../src/chapter3/4/loader.S}
\codecaption{为~ring 3~代码段准备的新栈(节自 chapter3/4/loader.S)}\label{CHpm_stackr3}

这个代码段的功能就是在第~11~行第~1~列打印一个~3~字。

其次,添加新段的描述符和选择子,并添加初始化代码:

\VerbatimInput[fontfamily=tt,fontsize=\footnotesize,frame=topline, framerule=0.4mm, numbers=left, numbersep=3pt, tabsize=2, firstline=29, lastline=31]{../src/chapter3/4/loader.S}
\VerbatimInput[fontfamily=tt,fontsize=\footnotesize,frame=topline, framerule=0.4mm, numbers=left, numbersep=3pt, tabsize=2, firstline=47, lastline=49]{../src/chapter3/4/loader.S}
\codecaption{为~ring 3~代码段和堆栈段添加的描述符和选择子(节自 chapter3/4/loader.S)}\label{CHpm_coder3_desc}

\VerbatimInput[fontfamily=tt,fontsize=\footnotesize,frame=lines, framerule=0.4mm, numbers=left, numbersep=3pt, tabsize=2, firstline=105, lastline=109]{../src/chapter3/4/loader.S}
\codecaption{初始化~ring 3~代码段和堆栈段描述符的代码(节自 chapter3/4/loader.S)}\label{CHpm_coder3_init}

我们注意到,其实~ring 3~下的代码段和~ring 0~下的代码段的代码部分是没有任何区别的,区别在于它们的代码段描述符和选择子中所标明的特权级。我们将~ring 3~下的目标代码段和堆栈段的段描述符属性中加上~\code{DA\_DPL3}~,表明它们的段描述符特权级均为~3~;在它们的段选择子属性中加上~\code{SA\_RPL3}~,表明它们的段选择子请求特权级均为~3~。以上这两点限制了这个段只能在~ring 3~下运行。

准备好了两个段,我们将这两个段的信息作为~SS, ESP, CS, EIP~的内容依次压栈,然后再执行~lret~长返回指令,就能跳转到~ring 3~下运行目标代码段了:

\VerbatimInput[fontfamily=tt,fontsize=\footnotesize,frame=lines, framerule=0.4mm, numbers=left, numbersep=3pt, tabsize=2, firstline=191, lastline=199]{../src/chapter3/4/loader.S}
\codecaption{hack RET~指令进行实际的跳转}\label{CHpm_coder3_init}

这样,我们就完成了从高特权级代码(ring 0)跳转到低特权级代码(ring 3)的过程。像往常那样使用~make, sudo make copy~编译,用虚拟机加载镜像运行结果如图~\ref{vb_run_9}~所示,程序在屏幕的第~11~行第~1~列打印出了一个红色的~'3'~字,然后进入死循环。

\FIG{hack RET~实现从高特权级到低特权级的跳转}{vb_run_9}{0.75\textwidth}

上面这个例子仅仅演示了如何从高特权级跳转到低特权级,而没有介绍如何转移回高特权级。直观来讲,我们只需将图~\ref{CHpm_coder3}~中的最后一条~JMP~指令替换成一条~CALL~调用门的指令即可。但是我们会看到,带有特权级转换的调用门转移并不是那么容易实现的。

下面我们来尝试一下直接访问调用门,首先将门描述符的~DPL~改为~3~以满足在~ring 3~代码段访问调用门的条件。

\begin{Command}
LABEL_CG_TEST:      Gate    SelectorCodeCG, 0, 0, (DA_386CGate + DA_DPL3)
\end{Command}

然后将图~\ref{CHpm_coder3}~中最后一条~JMP~指令替换成对调用门的~CALL~指令:

\begin{Command}
lcall   $(SelectorCGTest), $0  /* Call CODECG through call gate */
\end{Command}

然后编译运行一下,看看能否得到我们想要的结果?答案是否定的,我们看不到屏幕上打印出~'C'~字,得到与图~\ref{vb_main_4}~相同的结果。为什么呢?主要是因为在~CALL~调用门的时候发生了程序栈的切换,由于这个栈切换发生在~CALL~指令执行的过程中,CALL~指令要访问特殊的结构来得到新的栈信息,不像~RET~指令仅仅读取我们设置好的栈,如果访问出错就会产生一个异常,处理机无法处理这个异常就只好退出。我们下面一小节介绍在~CALL~调用门和~RET~的时候究竟发生了什么事?需要什么特殊结构的辅助才能实现带有特权级切换的调用门转移?

\subsection{栈切换和~TSS}

当使用调用门转移到不同特权级下的非一致代码段时,处理机总会自动的切换到目标特权级对应的栈。栈切换是为了避免高特权级的栈空间被滥用导致栈空间不足而崩溃,以及低特权级的程序非法修改或者干扰高特权级程序的栈内容。